\documentclass[border=10pt]{standalone}
\usepackage{tikz}
\usepackage{enumitem}
\usetikzlibrary{shapes.geometric, arrows}
\definecolor{primary}{RGB}{29,71,135}
\definecolor{secondary}{RGB}{94,96,98}
\tikzstyle{box} = [rectangle, rounded corners, text centered, draw=white, fill=primary,text=white,]
\tikzstyle{ghost} = [rectangle, rounded corners, text centered, draw=white]
\tikzstyle{arrow} = [latex-latex new, ultra thick, ->, >=stealth, draw=secondary]
\begin{document}
\begin{tikzpicture}[node distance=5cm, sibling distance=.5cm]
  % Draw nodes
  \node (db)[box,text width=6.5cm]{\textbf{CHIANTI Database} \\
        $\bullet$ First version released in 1996 \\
        $\bullet$ Currently 1.7 GB of ASCII text files \\
        $\bullet$ Thousands of transitions for ions of elements H through Zn
  };
  \node (spacer)[ghost, below of=db]{};
  \node (py)[box, left of=spacer, text width=7cm]{\textbf{ChiantiPy} \\
        $\bullet$ Developed by K. Dere beginning in 2003 \\
        $\bullet$ Released on SourceForge in 2009, GitHub in 2016 \\
        $\bullet$ Leverage full SciPy stack \\
        $\bullet$ Growing user community
  };
  \node (idl)[box, right of=spacer, text width=7cm]{\textbf{CHIANTI-IDL} \\
        $\bullet$ Original software distributed with database \\
        $\bullet$ Most popular interface to CHIANTI data \\
        $\bullet$ Freely distributed, but requires IDL license
  };
  % Draw arrows
  \draw [arrow] (db) -- (py);
  \draw [arrow] (db) -- (idl);
\end{tikzpicture}
\end{document}